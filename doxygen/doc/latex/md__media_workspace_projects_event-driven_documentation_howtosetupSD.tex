\+:warning\+: These instructions are the standard way to set-\/up the Y\+A\+R\+P/event-\/driven environment on an sd-\/card meant for the zcb or z-\/turn. If you already have a working sd-\/card you can simply copy the sd-\/card. You can find instructions for copying an sd-\/card at the \href{#how-copy-an-entire-sd-card-for-a-new-board}{\tt bottom}.

\section*{How to set-\/up an SD for a zynq board}

\+:warning\+: These instructions are only needed if your SD card has a fresh OS install and the {\ttfamily event-\/driven} library is not yet compiled. These instructions shuold be performed on the Z\+C\+B/z-\/turn and not your own laptop.

\subsection*{Set-\/up icub user}

As {\bfseries root}\+: 
\begin{DoxyCode}
adduser icub
groups icub
visudo
\end{DoxyCode}
 Add line\+: {\ttfamily icub A\+LL=(A\+LL\+:A\+LL) A\+LL}

\subsection*{Set up repositories}

As {\bfseries icub}\+:
\begin{DoxyItemize}
\item Follow the installation instructions putting projects into {\ttfamily $\sim$/projects} and installing into {\ttfamily $\sim$/install}. You will need to make these folders in the home directory of icub.
\end{DoxyItemize}

\subsubsection*{Note 1\+:}

As only basic {\ttfamily Y\+A\+RP} support is needed, not all dependencies are required to be installed, instead install only\+: 
\begin{DoxyCode}
sudo apt install [TODO]
\end{DoxyCode}
 \subsubsection*{Note 2\+:}

The newest Y\+A\+RP requires C\+Make$>$3.\+5, which is not installable via {\ttfamily apt} on the Debian 8.\+10 (jessie) distribution we have installed on the zynq. To upgrade C\+Make you need to install it via backports (reference\+: \href{https://backports.debian.org/Instructions}{\tt https\+://backports.\+debian.\+org/\+Instructions}). To do so\+:
\begin{DoxyItemize}
\item add to {\ttfamily /etc/apt/sources.list} the line below\+: ```bash deb \href{http://ftp.debian.org/debian}{\tt http\+://ftp.\+debian.\+org/debian} jessie-\/backports main ``` Then\+: 
\begin{DoxyCode}
sudo apt update
sudo apt -t jessie-backports install cmake
\end{DoxyCode}
 At this point you should be able to recompile Y\+A\+RP 3.\+0 and {\ttfamily event-\/driven} {\bfseries master} branch.
\end{DoxyItemize}

{\bfseries We can consider updating the Debian distribution of the zynq boards since the Debian 8.\+10 is no longer supported by Y\+A\+RP}

\subsubsection*{Note 3\+:}

When installing {\ttfamily event-\/driven} use the following options for cmake\+: 
\begin{DoxyCode}
cmake .. -DCMAKE\_INSTALL\_PREFIX=$INSTALL\_DIR -DBUILD\_HARDWAREIO=ON -DENABLE\_zynqgrabber=ON
\end{DoxyCode}
 \subsection*{Set up device drivers}

As {\bfseries icub}\+: 
\begin{DoxyCode}
sudo usermod -a -G i2c icub
sudo vim /lib/udev/rules.d/77-iit-hpu.rules
\end{DoxyCode}
 Add lines\+: 
\begin{DoxyCode}
SUBSYSTEM=="iit-hpu-class", GROUP="i2c"
\end{DoxyCode}
 Then 
\begin{DoxyCode}
sudo vim /etc/rc.local
\end{DoxyCode}
 Add lines\+: 
\begin{DoxyCode}
insmod /home/icub/iit-hpucore-dma.ko rx\_pn=1024 rx\_ps=8096 rx\_to=5000
\end{DoxyCode}
 \subsection*{Misc}

Check the device driver meta data\+: 
\begin{DoxyCode}
udevadm info -q all -a /dev/iit-hpu0
\end{DoxyCode}
 Check the device driver parameters\+: 
\begin{DoxyCode}
cat /sys/module/iit\_hpucore\_dma/parameters/ps
\end{DoxyCode}


\section*{How copy an entire sd-\/card for a new board}

\subsection*{P\+A\+R\+T\+I\+T\+I\+ON T\+HE N\+EW SD}


\begin{DoxyItemize}
\item Insert the new SD
\item {\ttfamily sudo gparted} ({\ttfamily sudo apt-\/get install gparted} if needed)
\item {\ttfamily gparted} G\+UI should detect the SD
\item Unmount the SD in gparted G\+UI (you cannot partition a mounted drive)
\item Create new partitions\+: 1. F\+A\+T32 name\+:B\+O\+OT 50\+MiB 2. E\+X\+T4 name\+:rootfs (max-\/250) 3. linux-\/swap name\+:swap 200\+MiB
\item Edit -\/$>$ apply all operations
\end{DoxyItemize}

\subsection*{C\+O\+PY T\+HE F\+I\+L\+ES}


\begin{DoxyItemize}
\item Insert old SD (mount the boot and filesystem partitions)
\item Copy B\+O\+OT (old) -\/$>$ B\+O\+OT (new) (use {\ttfamily /tmp} as a temporary location to store files if you cannot mount both SD cards simultaneously)
\item {\ttfamily sudo tar zcvf filesystem.\+tgz /media/\$username/rootfs} (from the old SD -\/ again do this in {\ttfamily /tmp})
\item {\ttfamily sudo sync} (ensure files are copied by flushing file writing queue)
\item {\ttfamily cd /media/\$username/rootfs} (on the new SD)
\item {\ttfamily sudo tar zxvf /tmp/filesystem.tgz -\/-\/strip-\/components=3}
\item {\ttfamily sudo sync} 
\end{DoxyItemize}