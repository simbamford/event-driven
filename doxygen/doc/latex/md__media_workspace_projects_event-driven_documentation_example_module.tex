You are ready to write your own code that uses the tools in the {\ttfamily event-\/driven} library to process events. An example module is provided in example\+\_\+module that you can use as the basis for writing a module that can be integrated into the {\ttfamily Y\+A\+RP} framework.

The module has the following functionality\+:


\begin{DoxyItemize}
\item Reading a {\ttfamily $\ast$.ini} file to load configuration options, and setting options via command line arguments.
\item Clean exit by capturing {\ttfamily ctrl+c} commands and allowing functions to close ports and clean memory.
\item Example {\ttfamily event-\/driven} ports for reading and writing events.
\item A synchronous thread, typically used to show a visualisation or status/debug messages at a readable rate.
\item An asynchronous thread, used to read events at sub-\/millisecond rates and perform processing as required.
\item An example C\+Make file for installing the generated binaries in the install location of {\ttfamily event-\/driven}, as well as installation of the configuration and application files in locations required by {\ttfamily Y\+A\+RP}.
\end{DoxyItemize}

\subsection*{How to use the example-\/module}

First, copy the example files to the new location of your project, {\itshape e.\+g.} the same folder where you have cloned {\ttfamily event-\/driven}. Assuming a{\ttfamily $<$path\+\_\+to\+\_\+projects$>$} directory\+: 
\begin{DoxyCode}
cp -r <path\_to\_projects>/event-driven/documentation/example-module <path\_to\_projects>
\end{DoxyCode}


The project can be compiled with modifications\+: 
\begin{DoxyCode}
cd <path\_to\_projects>/example-module
mkdir build && cd build
cmake .. -DCMAKE\_INSTALL\_PREFIX=$INSTALL\_DIR
make install -j4
\end{DoxyCode}


However, some modifications should be made to personalise the project. The first step would be to change the folder, file and project name\+: 
\begin{DoxyCode}
cd <path\_to\_projects>
mv example-module <my-module-name>
mv <my-module-name>/example-module.cpp <my-module-name>/<my-module-name>.cpp
mv example-module.ini <my-module-name>.ini
mv app\_example-module.xml app\_<my-module-name>.xml
nano <my-module-name> CMakeLists.txt
\end{DoxyCode}
 On \href{https://github.com/robotology/event-driven/blob/99a1f941141b33266900e034d3e7789d55fd0d99/documentation/example-module/CMakeLists.txt#L5}{\tt line 5} change the project name to {\ttfamily $<$my-\/module-\/name$>$}, then save ({\ttfamily ctrl+o}) and exit ({\ttfamily ctrl+x}). 
\begin{DoxyCode}
nano <my-module-name>/<my-module-name>.cpp
\end{DoxyCode}
 On line \href{https://github.com/robotology/event-driven/blob/99a1f941141b33266900e034d3e7789d55fd0d99/documentation/example-module/example-module.cpp#L7}{\tt 7}, \href{https://github.com/robotology/event-driven/blob/99a1f941141b33266900e034d3e7789d55fd0d99/documentation/example-module/example-module.cpp#L19}{\tt 19},\href{https://github.com/robotology/event-driven/blob/99a1f941141b33266900e034d3e7789d55fd0d99/documentation/example-module/example-module.cpp#L24}{\tt 24}, and \href{https://github.com/robotology/event-driven/blob/99a1f941141b33266900e034d3e7789d55fd0d99/documentation/example-module/example-module.cpp#L124}{\tt 124} change the class, constructor and declaration to {\ttfamily $<$my-\/module-\/name$>$}, then save ({\ttfamily ctrl+o}) and exit ({\ttfamily ctrl+x}).

Change the first line of the default configuration file and the lines that refer to {\ttfamily example-\/module} in the example {\ttfamily yarpmanager} application. For example\+:


\begin{DoxyCode}
nano <my-module-name>/<my-module-name>.ini
nano <my-module-name>/app\_<my-module-name>.xml
\end{DoxyCode}


or if you like\+:


\begin{DoxyCode}
sed -i 's/example-module/<my-module-name>/g' <my-module-name>.xml
\end{DoxyCode}


The module should now be personalised to your processing task.

If you like, you can import the project into your favourite I\+DE. To do so, {\itshape e.\+g.} for \href{https://www.qt.io/product}{\tt Qt\+Creator} {\itshape open a new project} by selecting {\ttfamily $<$path\+\_\+to\+\_\+projects$>$/$<$my-\/module-\/name$>$/\+Cmake\+Lists.txt}. Select the kits {\ttfamily release} and {\ttfamily debug} modifying the build directory to {\ttfamily $<$path\+\_\+to\+\_\+projects$>$/$<$my-\/module-\/name$>$/build} and {\ttfamily $<$path\+\_\+to\+\_\+projects$>$/$<$my-\/module-\/name$>$/build-\/debug} respectively. You should now be able to edit, compile, run and debug the module from within Qt\+Creator. 